
\begin{table}[ht!!]
\caption{格差縮小フレームが金融緩和選好に与える効果に対するイデオロギーの条件付け(統制変数無;金融緩和選好とイデオロギー変数の「わからない」回答は分析から除外)}
\begin{center}
\begin{scriptsize}
\begin{tabular}{l D{.}{.}{3.6} D{.}{.}{3.6} D{.}{.}{3.6} D{.}{.}{3.6} }
\toprule
 & \multicolumn{1}{c}{自己申告} & \multicolumn{1}{c}{政党支持} & \multicolumn{1}{c}{外交安全保障} & \multicolumn{1}{c}{権利機会平等} \\
\midrule
(定数項)           & 1.053^{***} & 0.972^{***} & 0.931^{***} & 0.927^{***} \\
                & (0.114)     & (0.105)     & (0.132)     & (0.146)     \\
2X.経済成長&格差縮小    & -0.038      & -0.021      & 0.039       & 0.042       \\
                & (0.154)     & (0.147)     & (0.173)     & (0.187)     \\
イデオロギー          & 0.240^{**}  & 0.291^{*}   & 0.623^{***} & 0.017       \\
                & (0.090)     & (0.146)     & (0.150)     & (0.144)     \\
イデオロギー×2X.成長&格差 & -0.187      & -0.073      & -0.289      & -0.160      \\
                & (0.129)     & (0.199)     & (0.201)     & (0.178)     \\
\midrule
R$^2$           & 0.029       & 0.023       & 0.136       & 0.009       \\
Adj. R$^2$      & 0.019       & 0.015       & 0.124       & -0.006      \\
Num. obs.       & 289         & 337         & 210         & 210         \\
RMSE            & 1.271       & 1.264       & 1.243       & 1.332       \\
\bottomrule
\multicolumn{5}{l}{\tiny{\parbox{1.0\linewidth}{\vspace{2pt}$^{***}p<0.001$, $^{**}p<0.01$, $^*p<0.05$, $^{\dagger}p<0.1$. \\ 最小二乗法による重回帰分析.( )内はロバスト標準誤差(HC2). }}}
\end{tabular}
\end{scriptsize}
\label{idetab_h2x_n1}
\end{center}
\end{table}
