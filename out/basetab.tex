
\begin{table}[ht!!]
\caption{実験情報刺激が金融緩和選好に与える効果(重回帰分析)}
\begin{center}
\begin{footnotesize}
\begin{tabular}{l D{)}{)}{13)7} }
\toprule
 & \multicolumn{1}{c}{基本モデル} \\
\midrule
(定数項)            & 1.131 \; (0.204)^{***}      \\
1.経済成長           & 0.130 \; (0.129)            \\
2.経済成長&貧困削減      & 0.287 \; (0.127)^{*}        \\
3.経済成長&格差縮小      & 0.077 \; (0.121)            \\
4.経済成長&学者賛成      & 0.137 \; (0.122)            \\
5.経済成長&貧困削減&学者賛成 & 0.368 \; (0.120)^{**}       \\
政治知識             & 0.235 \; (0.143)            \\
性別(女性)           & -0.317 \; (0.079)^{***}     \\
年齢               & -0.008 \; (0.004)^{*}       \\
居住年数             & -0.058 \; (0.030)^{\dagger} \\
持ち家              & -0.002 \; (0.081)           \\
教育:短大/高専/専門学校    & 0.145 \; (0.124)            \\
教育:大卒以上          & 0.102 \; (0.101)            \\
就労               & -0.005 \; (0.082)           \\
婚姻               & -0.106 \; (0.111)           \\
子ども              & 0.222 \; (0.112)^{*}        \\
\midrule
R$^2$            & 0.045                       \\
Adj. R$^2$       & 0.032                       \\
Num. obs.        & 1123                        \\
RMSE             & 1.203                       \\
\bottomrule
\multicolumn{2}{l}{\tiny{\parbox{.4\linewidth}{\vspace{2pt}$^{***}p<0.001$, $^{**}p<0.01$, $^*p<0.05$, $^{dagger}p<0.1$. \\ 最小二乗法による重回帰分析、ロバスト標準誤差使用. }}}
\end{tabular}
\end{footnotesize}
\label{basetab}
\end{center}
\end{table}
