% latex table generated in R 3.6.3 by xtable 1.8-4 package
% Tue Jun  9 23:30:00 2020
\begin{table}[ht]
\centering
\caption{実験群ごとのマニピュレーションチェック違反者の分布(違反者は回答が表示された/されていない情報と完全に一致しない被験者)} 
\label{mchecktab_2}
\begin{tabular}{lll}
  \hline
 & 非違反者数 & 違反者数 \\ 
  \hline
統制群 & 143 &  41 \\ 
  1.経済成長 &  94 &  91 \\ 
  2.経済成長&貧困削減 &  37 & 148 \\ 
  2X.経済成長&格差縮小 &  39 & 157 \\ 
  3.経済成長&学者賛成 &  43 & 139 \\ 
  4.経済成長&貧困&学者 &  26 & 165 \\ 
   \hline
\end{tabular}
\end{table}
