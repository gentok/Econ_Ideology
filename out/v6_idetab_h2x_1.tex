
\begin{table}[ht!!]
\caption{格差縮小フレームが金融緩和選好に与える効果に対するイデオロギーの条件付け(統制変数無;金融緩和選好とイデオロギー変数の「わからない」回答には0を代入)}
\begin{center}
\begin{scriptsize}
\begin{tabular}{l D{.}{.}{3.6} D{.}{.}{3.10} D{.}{.}{3.6} D{.}{.}{3.6} }
\toprule
 & \multicolumn{1}{c}{自己申告} & \multicolumn{1}{c}{政党支持} & \multicolumn{1}{c}{外交安全保障} & \multicolumn{1}{c}{権利機会平等} \\
\midrule
(定数項)           & 0.883^{***} & 0.861^{***}     & 0.899^{***} & 0.899^{***} \\
                & (0.098)     & (0.098)         & (0.091)     & (0.098)     \\
2X.経済成長&格差縮小    & 0.001       & -0.020          & 0.002       & -0.022      \\
                & (0.131)     & (0.133)         & (0.123)     & (0.128)     \\
イデオロギー          & 0.217^{*}   & 0.256^{\dagger} & 0.465^{***} & -0.106      \\
                & (0.091)     & (0.135)         & (0.101)     & (0.100)     \\
イデオロギー×2X.成長&格差 & -0.169      & 0.006           & -0.279^{*}  & -0.013      \\
                & (0.130)     & (0.182)         & (0.137)     & (0.126)     \\
\midrule
R$^2$           & 0.019       & 0.024           & 0.078       & 0.011       \\
Adj. R$^2$      & 0.012       & 0.016           & 0.071       & 0.003       \\
Num. obs.       & 381         & 381             & 381         & 381         \\
RMSE            & 1.241       & 1.238           & 1.203       & 1.246       \\
\bottomrule
\multicolumn{5}{l}{\tiny{\parbox{1.0\linewidth}{\vspace{2pt}$^{***}p<0.001$, $^{**}p<0.01$, $^*p<0.05$, $^{\dagger}p<0.1$. \\ 最小二乗法による重回帰分析.( )内はロバスト標準誤差(HC2). }}}
\end{tabular}
\end{scriptsize}
\label{idetab_h2x_1}
\end{center}
\end{table}
