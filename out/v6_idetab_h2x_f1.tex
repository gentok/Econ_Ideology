
\begin{table}[ht!!]
\caption{格差縮小フレームが金融緩和選好に与える効果に対するイデオロギーの条件付け(統制変数無;マニピュレーションチェックに違反した回答者を分析から除外)}
\begin{center}
\begin{scriptsize}
\begin{tabular}{l D{.}{.}{3.6} D{.}{.}{3.10} D{.}{.}{3.6} D{.}{.}{3.6} }
\toprule
 & \multicolumn{1}{c}{自己申告} & \multicolumn{1}{c}{政党支持} & \multicolumn{1}{c}{外交安全保障} & \multicolumn{1}{c}{権利機会平等} \\
\midrule
(定数項)           & 0.811^{***} & 0.793^{***}     & 0.833^{***} & 0.820^{***} \\
                & (0.101)     & (0.101)         & (0.094)     & (0.101)     \\
2X.経済成長&格差縮小    & 0.047       & 0.011           & 0.046       & 0.035       \\
                & (0.139)     & (0.142)         & (0.131)     & (0.137)     \\
イデオロギー          & 0.234^{*}   & 0.244^{\dagger} & 0.464^{***} & -0.148      \\
                & (0.095)     & (0.140)         & (0.106)     & (0.108)     \\
イデオロギー×2X.成長&格差 & -0.170      & 0.055           & -0.310^{*}  & 0.049       \\
                & (0.136)     & (0.194)         & (0.147)     & (0.138)     \\
\midrule
R$^2$           & 0.024       & 0.026           & 0.078       & 0.012       \\
Adj. R$^2$      & 0.015       & 0.017           & 0.069       & 0.003       \\
Num. obs.       & 333         & 333             & 333         & 333         \\
RMSE            & 1.241       & 1.240           & 1.207       & 1.249       \\
\bottomrule
\multicolumn{5}{l}{\tiny{\parbox{1.0\linewidth}{\vspace{2pt}$^{***}p<0.001$, $^{**}p<0.01$, $^*p<0.05$, $^{\dagger}p<0.1$. \\ 最小二乗法による重回帰分析.( )内はロバスト標準誤差(HC2). }}}
\end{tabular}
\end{scriptsize}
\label{idetab_h2x_f1}
\end{center}
\end{table}
